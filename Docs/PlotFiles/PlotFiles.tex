\documentclass[12pt]{article}
\setlength{\textheight}{9.80in}
\setlength{\textwidth}{6.40in}
\setlength{\oddsidemargin}{0.0mm}
\setlength{\evensidemargin}{1.0mm}
\setlength{\topmargin}{-0.6in}
\setlength{\parindent}{0.2in}
\setlength{\parskip}{1.5ex}
\newtheorem{defn}{Definition}
\renewcommand{\baselinestretch}{1.2}

\begin{document}

\bibliographystyle{prsty}
\thispagestyle{empty}


\title{The $PlotFiles$}

\author{C. Godsalve \\
   email: seagods@hotmail.com}

\maketitle

\tableofcontents


\section{Introduction}

You will not be surprised to find that the "$PlotFiles$" are a for 
plotting graphs and visualising data. There are a lot of programes
"out there" to do this of course. However, I didn't like some, and couldn't
afford others. At the same time I was learning a bit about OpenGL,
(see www.opengl.org). One result was the $PlotFiles$. I have taken the route
of writing a series of stand-alone programs rather than attempting 
an integrated graphics package. The programs are meant to be run
on unix-like systems and are meant to be run via command line.
(They can be used point-and-click.) If anyone likes them, and can get
them to work on other operating systems, please contribute!

What are the $Plotfiles$? So far they consist of the following.
The simplest is $PlotIt$. This plots multiple $y=f(x)$ type data
as points or lines. Next comes $PlotIt2$ which plots multiple parametric
space curves of the form $(x=x(t),y=y(t), z=z(t))$ where $t$ is the parameter.

One step up, and we have $PlotCont$ which visualises a function
 $z=f(x,y)$ as a flat contour plot, where $z$ values are represented
by colour as well as contours. This plots only one function $z(x,y)$.
To plot multiple functions of the type $z=f(x,y)$ as curved two 
dimensional surfaces embeded in three dimensions. There are 
two other programs which are similar to $PlotCon$. One is $PlotWorld$
which is a specialised version, where $x$ and $y$ are longitude
and latitude between $\pm 180^\circ$ and $\pm 90^\circ$. We 
overlay the plot with NOAA World Vector Shoreline data. 
Another specialisation of $PlotCont$ is $PlotPol$ which
 does a polar plot of the form $z=f(\theta,\phi)$ where
 $ 0 \le \theta \le \pi/2$ and  $ 0 \le \phi \l  2\pi$.

Now, $PlotCont$ plots 1D curved contours in a flat 2D space. 
The value of $z=f(x,y)$ is represented by colour.
A generalisation of this is $PlotVol$ which plots 2D curved
 surface contours embedded in a flat 3D space. The value
of $w=f(x,y,z)$ is represented by colour. Unlike PlotCont,
 where we can view the plot from the "outside" of the 2D
flat surface, we are "in" the 3D flat space. Lighting and
transparency are used to help the visualisation.

Lastly we have $PlotGlobe$ which wraps a bitmap round
a spherical OpenGL quadric. This is a place-holder for
further work. In future I wish to make some time dependendent
 visualisation programs. 

In all the 2D programs you can move the plot left
and right, and up and down, using the <Ctrl> <arrow>.
The unmodified up and down arrows zoom in and out.
In the 3D programs, you can move around in 3D.
All the progams have bitmap output and postscript
output (via gl2ps www.guez.org/gl2ps). There are other
controls. All the programs can be quit via <Esc>, and
 have the same help screen (toggled with <F1>.) There are other
controls. All the programs can be quit via <Esc>, and
 have the same help screen (toggled with <F1>).


At this point I must tell you that I have managed to lose the
original tex file. I shall rewrite it soon, but for
now I leave you with the surviving pdf file. Here it
explains the dependencies (gltt, openGL or Mesa, SDL, and truetype).
I must warn you that the size of some of the images means that
it is very slow. I shall make some videos which will be more helpful
soon.




\end{document}

